%% Generated by Sphinx.
\def\sphinxdocclass{report}
\documentclass[letterpaper,10pt,english]{sphinxmanual}
\ifdefined\pdfpxdimen
   \let\sphinxpxdimen\pdfpxdimen\else\newdimen\sphinxpxdimen
\fi \sphinxpxdimen=.75bp\relax

\PassOptionsToPackage{warn}{textcomp}
\usepackage[utf8]{inputenc}
\ifdefined\DeclareUnicodeCharacter
% support both utf8 and utf8x syntaxes
  \ifdefined\DeclareUnicodeCharacterAsOptional
    \def\sphinxDUC#1{\DeclareUnicodeCharacter{"#1}}
  \else
    \let\sphinxDUC\DeclareUnicodeCharacter
  \fi
  \sphinxDUC{00A0}{\nobreakspace}
  \sphinxDUC{2500}{\sphinxunichar{2500}}
  \sphinxDUC{2502}{\sphinxunichar{2502}}
  \sphinxDUC{2514}{\sphinxunichar{2514}}
  \sphinxDUC{251C}{\sphinxunichar{251C}}
  \sphinxDUC{2572}{\textbackslash}
\fi
\usepackage{cmap}
\usepackage[T1]{fontenc}
\usepackage{amsmath,amssymb,amstext}
\usepackage{babel}



\usepackage{times}
\expandafter\ifx\csname T@LGR\endcsname\relax
\else
% LGR was declared as font encoding
  \substitutefont{LGR}{\rmdefault}{cmr}
  \substitutefont{LGR}{\sfdefault}{cmss}
  \substitutefont{LGR}{\ttdefault}{cmtt}
\fi
\expandafter\ifx\csname T@X2\endcsname\relax
  \expandafter\ifx\csname T@T2A\endcsname\relax
  \else
  % T2A was declared as font encoding
    \substitutefont{T2A}{\rmdefault}{cmr}
    \substitutefont{T2A}{\sfdefault}{cmss}
    \substitutefont{T2A}{\ttdefault}{cmtt}
  \fi
\else
% X2 was declared as font encoding
  \substitutefont{X2}{\rmdefault}{cmr}
  \substitutefont{X2}{\sfdefault}{cmss}
  \substitutefont{X2}{\ttdefault}{cmtt}
\fi


\usepackage[Bjarne]{fncychap}
\usepackage{sphinx}

\fvset{fontsize=\small}
\usepackage{geometry}


% Include hyperref last.
\usepackage{hyperref}
% Fix anchor placement for figures with captions.
\usepackage{hypcap}% it must be loaded after hyperref.
% Set up styles of URL: it should be placed after hyperref.
\urlstyle{same}


\usepackage{sphinxmessages}
\setcounter{tocdepth}{0}



\title{Patent2Net Documentation}
\date{Jan 01, 2021}
\release{3.1.10-dev6}
\author{The Patent2Net Developers}
\newcommand{\sphinxlogo}{\vbox{}}
\renewcommand{\releasename}{Release}
\makeindex
\begin{document}

\pagestyle{empty}
\sphinxmaketitle
\pagestyle{plain}
\sphinxtableofcontents
\pagestyle{normal}
\phantomsection\label{\detokenize{index::doc}}


../README.rst


\chapter{Getting started}
\label{\detokenize{index:getting-started}}

\section{Setup}
\label{\detokenize{setup:setup}}\label{\detokenize{setup:id1}}\label{\detokenize{setup::doc}}
This part of the documentation covers the installation of Patent2Net.
The first step to using any software package is getting it properly installed.
Please read this section carefully.

After successfully installing the software, you might want to
follow up with its {\hyperref[\detokenize{configure:configuration}]{\sphinxcrossref{\DUrole{std,std-ref}{Configuration}}}}.


\subsection{Installation}
\label{\detokenize{setup:installation}}\begin{itemize}
\item {} 
Install docker desktop

\item {} 
download the docker\sphinxhyphen{}install branch (\sphinxurl{https://github.com/Patent2net/P2N-v3/tree/docker\_install})

\item {} 
Use the Install\_P2N.bat and enjoy P2N via docker image

\item {} 
See Install.bat in the elatickibana directory for the full architecture (BETA)

\end{itemize}


\subsection{Old ways}
\label{\detokenize{setup:old-ways}}

\subsubsection{Install from source}
\label{\detokenize{setup:install-from-source}}\begin{itemize}
\item {} 
Download Anaconda 3 (big installation, \sphinxurl{https://www.anaconda.com/distribution/}) OR Miniconda \sphinxurl{https://docs.conda.io/en/latest/miniconda.html}

\item {} 
Download P2N latest \sphinxurl{https://github.com/Patent2net/P2N-v3}

\item {} 
Unzip to a the P2N directory (please choose a safe path: the root of your second hard drive (d:P2N) for instance is the prefered choice)

\item {} 
Launch (double click) installP2N.bat (let it work; do not close windows; that process can be quite long depending on you machine and network access)

\item {} 
Copy or create cles\sphinxhyphen{}epo.txt file (be aware must be ANSI encode caracters!). See \sphinxurl{https://docs.ip-tools.org/patent2net/configure.html}

\item {} 
Formulate the desired {\color{red}\bfseries{}*}.cql file in ./REQUESTS/ directory. See \sphinxurl{https://docs.ip-tools.org/patent2net/usage/index.html}

\item {} 
Double click runP2N.bat and have fun!

\end{itemize}

Nexr seems broken
\sphinxhyphen{} Source releases are available from GitHub: \sphinxurl{https://github.com/Patent2net/P2N/releases}
\sphinxhyphen{} Install a specific version:

\begin{sphinxVerbatim}[commandchars=\\\{\}]
\PYG{n}{pip} \PYG{n}{install} \PYG{l+s+s1}{\PYGZsq{}}\PYG{l+s+s1}{https://github.com/Patent2net/P2N/archive/3.0.0\PYGZhy{}dev6.tar.gz}\PYG{l+s+s1}{\PYGZsq{}} \PYG{o}{\PYGZhy{}}\PYG{o}{\PYGZhy{}}\PYG{n}{upgrade}
\end{sphinxVerbatim}
\begin{itemize}
\item {} 
Install the current development version:

\begin{sphinxVerbatim}[commandchars=\\\{\}]
\PYG{n}{pip} \PYG{n}{install} \PYG{l+s+s1}{\PYGZsq{}}\PYG{l+s+s1}{https://github.com/Patent2net/P2N/archive/develop.tar.gz}\PYG{l+s+s1}{\PYGZsq{}} \PYG{o}{\PYGZhy{}}\PYG{o}{\PYGZhy{}}\PYG{n}{upgrade}
\end{sphinxVerbatim}

\end{itemize}


\subsubsection{Install binary package (Old features)}
\label{\detokenize{setup:install-binary-package-old-features}}\begin{itemize}
\item {} 
\begin{DUlineblock}{0em}
\item[] Binary releases are available at
\item[] \sphinxurl{http://patent2netv2.vlab4u.info/dokuwiki/doku.php?id=user\_manual:download\_install}
\end{DUlineblock}

\end{itemize}


\subsection{Appendix}
\label{\detokenize{setup:appendix}}

\subsubsection{Install ImageMagick}
\label{\detokenize{setup:install-imagemagick}}
On Debian Linux:

\begin{sphinxVerbatim}[commandchars=\\\{\}]
\PYG{n}{apt} \PYG{n}{install} \PYG{n}{imagemagick}
\end{sphinxVerbatim}

On Windows:

\begin{sphinxVerbatim}[commandchars=\\\{\}]
\PYG{n}{https}\PYG{p}{:}\PYG{o}{/}\PYG{o}{/}\PYG{n}{www}\PYG{o}{.}\PYG{n}{imagemagick}\PYG{o}{.}\PYG{n}{org}\PYG{o}{/}\PYG{n}{script}\PYG{o}{/}\PYG{n}{download}\PYG{o}{.}\PYG{n}{php}\PYG{c+c1}{\PYGZsh{}windows}
\end{sphinxVerbatim}


\subsubsection{Install Patent2Net on Linux}
\label{\detokenize{setup:install-patent2net-on-linux}}
If you’re using Ubuntu or Debian distributions, make sure to have these prerequisites installed:

\begin{sphinxVerbatim}[commandchars=\\\{\}]
\PYG{n}{sudo} \PYG{n}{apt}\PYG{o}{\PYGZhy{}}\PYG{n}{get} \PYG{n}{install} \PYG{n}{python}\PYG{o}{\PYGZhy{}}\PYG{n}{pip} \PYG{n}{build}\PYG{o}{\PYGZhy{}}\PYG{n}{essential} \PYG{n}{python}\PYG{o}{\PYGZhy{}}\PYG{n}{dev} \PYG{n}{libjpeg}\PYG{o}{\PYGZhy{}}\PYG{n}{dev} \PYG{n}{libxml2}\PYG{o}{\PYGZhy{}}\PYG{n}{dev} \PYG{n}{libfreetype6}\PYG{o}{\PYGZhy{}}\PYG{n}{dev} \PYG{n}{libpng}\PYG{o}{\PYGZhy{}}\PYG{n}{dev}
\end{sphinxVerbatim}

Install pygraphviz on Mac OS X:

\begin{sphinxVerbatim}[commandchars=\\\{\}]
\PYG{n}{pip} \PYG{n}{install} \PYG{o}{\PYGZhy{}}\PYG{o}{\PYGZhy{}}\PYG{n}{install}\PYG{o}{\PYGZhy{}}\PYG{n}{option}\PYG{o}{=}\PYG{l+s+s2}{\PYGZdq{}}\PYG{l+s+s2}{\PYGZhy{}\PYGZhy{}include\PYGZhy{}path=/opt/local/include}\PYG{l+s+s2}{\PYGZdq{}} \PYG{o}{\PYGZhy{}}\PYG{o}{\PYGZhy{}}\PYG{n}{install}\PYG{o}{\PYGZhy{}}\PYG{n}{option}\PYG{o}{=}\PYG{l+s+s2}{\PYGZdq{}}\PYG{l+s+s2}{\PYGZhy{}\PYGZhy{}library\PYGZhy{}path=/opt/local/lib}\PYG{l+s+s2}{\PYGZdq{}} \PYG{l+s+s2}{\PYGZdq{}}\PYG{l+s+s2}{pygraphviz==1.3.1}\PYG{l+s+s2}{\PYGZdq{}}
\end{sphinxVerbatim}


\subsubsection{Install Patent2Net on Windows}
\label{\detokenize{setup:install-patent2net-on-windows}}\begin{itemize}
\item {} 
Install Graphviz (I used 2.38)

\item {} 
Anaconda
\begin{itemize}
\item {} 
Starting from Anaconda 2.7 installation (32 or 64 bits)

\item {} 
To install Anaconda go to \sphinxurl{https://www.continuum.io/downloads}

\item {} 
Use Anaconda 4.2.0 or superior version

\item {} 
Use Python 2.7 version

\end{itemize}

\item {} 
Then:

\begin{sphinxVerbatim}[commandchars=\\\{\}]
\PYG{n}{pip} \PYG{n}{install} \PYG{o}{\PYGZhy{}}\PYG{o}{\PYGZhy{}}\PYG{n}{upgrade} \PYG{n}{requests}
\PYG{n}{pip} \PYG{n}{install} \PYG{o}{\PYGZhy{}}\PYG{o}{\PYGZhy{}}\PYG{n}{upgrade} \PYG{n}{networkx}
\PYG{n}{pip} \PYG{n}{install} \PYG{o}{\PYGZhy{}}\PYG{o}{\PYGZhy{}}\PYG{n}{upgrade} \PYG{n}{setuptools}
\end{sphinxVerbatim}

\item {} 
Download the epo\_ops directory of \sphinxurl{https://github.com/55minutes/python-epo-ops-client} in …Anaconda2Libsite\sphinxhyphen{}packages

\item {} 
In IPython: import epo\_ops

\item {} 
Download curses \sphinxurl{http://www.lfd.uci.edu/~gohlke/pythonlibs/\#curses} followed by
\sphinxcode{\sphinxupquote{pip install curses\sphinxhyphen{}2.2\sphinxhyphen{}cp2.7(...).whl}}

\item {} 
Download \sphinxurl{https://bitbucket.org/taynaud/python-louvain} followed by
\sphinxcode{\sphinxupquote{python setup.py install}}

\item {} 
Install pygraphviz (for 64 bits use: \sphinxurl{http://www.lfd.uci.edu/~gohlke/pythonlibs/\#pygraphviz})

\item {} 
\sphinxcode{\sphinxupquote{pip install \sphinxhyphen{}\sphinxhyphen{}upgrade jinja2}}

\end{itemize}


\section{Configuration}
\label{\detokenize{configure:configuration}}\label{\detokenize{configure:id1}}\label{\detokenize{configure::doc}}
This part of the documentation covers the configuration of Patent2Net.
The second step to using any software package is getting it properly configured.
Please read this section carefully.


\subsection{docker version via web interface}
\label{\detokenize{configure:docker-version-via-web-interface}}
Follow the get started section: \sphinxurl{http://localhost:5000/get\_started}

After successfully configuring the software, you might want to
follow up reading about its {\hyperref[\detokenize{usage/index:usage}]{\sphinxcrossref{\DUrole{std,std-ref}{Usage}}}}.


\subsection{OPS credentials}
\label{\detokenize{configure:ops-credentials}}
Patent2Net needs your personal credentials for accessing the OPS API.
You have to provide them once as they are stored into the file
\sphinxcode{\sphinxupquote{cles\sphinxhyphen{}epo.txt}} in the current working directory.

There is a convenience command for initializing Patent2Net with your OPS credentials:

\begin{sphinxVerbatim}[commandchars=\\\{\}]
\PYG{n}{p2n} \PYG{n}{ops} \PYG{n}{init} \PYG{o}{\PYGZhy{}}\PYG{o}{\PYGZhy{}}\PYG{n}{key}\PYG{o}{=}\PYG{n}{ScirfedyifJiashwOckNoupNecpainLo} \PYG{o}{\PYGZhy{}}\PYG{o}{\PYGZhy{}}\PYG{n}{secret}\PYG{o}{=}\PYG{n}{degTefyekDevgew1}
\end{sphinxVerbatim}

\begin{sphinxadmonition}{note}{Note:}
The Patent2Net scripts expect to find the file \sphinxcode{\sphinxupquote{cles\sphinxhyphen{}epo.txt}} in the projects’ root folder.
\end{sphinxadmonition}


\subsection{OPS registration}
\label{\detokenize{configure:ops-registration}}
Please follow the \sphinxhref{http://patent2netv2.vlab4u.info/dokuwiki/doku.php?id=user\_manual:download\_install\#register\_the\_use\_of\_p2n}{Register P2N with OPS} documentation to install your
OPS OAuth credentials in the “\sphinxcode{\sphinxupquote{cles\sphinxhyphen{}epo.txt}}” file in the root directory.


\section{Usage}
\label{\detokenize{usage/index:usage}}\label{\detokenize{usage/index:id1}}\label{\detokenize{usage/index::doc}}
\begin{sphinxShadowBox}
\begin{itemize}
\item {} 
\phantomsection\label{\detokenize{usage/index:id2}}{\hyperref[\detokenize{usage/index:introduction}]{\sphinxcrossref{Introduction}}}

\item {} 
\phantomsection\label{\detokenize{usage/index:id3}}{\hyperref[\detokenize{usage/index:operation}]{\sphinxcrossref{Operation}}}

\end{itemize}
\end{sphinxShadowBox}


\bigskip\hrule\bigskip



\subsection{Introduction}
\label{\detokenize{usage/index:introduction}}
For using Patent2Net, you should be reasonably familiar with the CQL query language
used for submitting queries to the Open Patent Services API.

Please follow up with these resources to learn something about CQL:
\begin{itemize}
\item {} 
\sphinxhref{http://patent2netv2.vlab4u.info/dokuwiki/doku.php?id=user\_manual:patent\_search}{CQL query howto} by the Patent2Net project

\item {} 
The \sphinxhref{http://documents.epo.org/projects/babylon/eponet.nsf/0/F3ECDCC915C9BCD8C1258060003AA712/\$FILE/ops\_v3.2\_documentation\%20\_version\_1.3.4\_en.pdf}{OPS Reference Guide}, see pages 56 ff. and pages 130 ff.

\item {} 
The \sphinxhref{https://www.loc.gov/standards/sru/cql/}{Contextual Query Language Specification}

\end{itemize}


\subsection{Operation}
\label{\detokenize{usage/index:operation}}
Patent2Net offers two modes of operation:


\subsubsection{Classic mode}
\label{\detokenize{usage/classic:classic-mode}}\label{\detokenize{usage/classic::doc}}
\begin{sphinxShadowBox}
\begin{itemize}
\item {} 
\phantomsection\label{\detokenize{usage/classic:id1}}{\hyperref[\detokenize{usage/classic:request-description}]{\sphinxcrossref{Request description}}}

\item {} 
\phantomsection\label{\detokenize{usage/classic:id2}}{\hyperref[\detokenize{usage/classic:legacy-interface}]{\sphinxcrossref{Legacy interface}}}

\item {} 
\phantomsection\label{\detokenize{usage/classic:id3}}{\hyperref[\detokenize{usage/classic:modern-interface}]{\sphinxcrossref{Modern interface}}}

\item {} 
\phantomsection\label{\detokenize{usage/classic:id4}}{\hyperref[\detokenize{usage/classic:synopsis}]{\sphinxcrossref{Synopsis}}}

\end{itemize}
\end{sphinxShadowBox}


\bigskip\hrule\bigskip



\paragraph{Request description}
\label{\detokenize{usage/classic:request-description}}
Patent2Net needs a \sphinxcode{\sphinxupquote{requete.cql}} file for operating in classic mode.
It acts as a request description and contains various parameters you might want to have a look at.

The most important ones are the CQL query to be submitted to OPS and the output directory where
data is stored.

Example:

\begin{sphinxVerbatim}[commandchars=\\\{\}]
\PYG{c+c1}{\PYGZsh{} CQL query expression}
\PYG{n}{request}\PYG{p}{:} \PYG{n}{ta}\PYG{o}{=}\PYG{l+s+s2}{\PYGZdq{}}\PYG{l+s+s2}{filter*}\PYG{l+s+s2}{\PYGZdq{}} \PYG{o+ow}{and} \PYG{n}{ta}\PYG{o}{=}\PYG{l+s+s2}{\PYGZdq{}}\PYG{l+s+s2}{drink* water}\PYG{l+s+s2}{\PYGZdq{}} \PYG{n}{AND} \PYG{p}{(}\PYG{n}{pn} \PYG{o}{=} \PYG{n}{U} \PYG{o+ow}{not} \PYG{p}{(}\PYG{n}{pn} \PYG{o}{=} \PYG{n}{UA} \PYG{o+ow}{or} \PYG{n}{pn} \PYG{o}{=} \PYG{n}{US} \PYG{o+ow}{or} \PYG{n}{pn} \PYG{o}{=} \PYG{n}{UY}\PYG{p}{)}\PYG{p}{)}

\PYG{c+c1}{\PYGZsh{} Output directory}
\PYG{n}{DataDirectory}\PYG{p}{:} \PYG{n}{Water}
\end{sphinxVerbatim}

You can find some blueprints in the \sphinxcode{\sphinxupquote{/RequestsSets}} directory.


\paragraph{Legacy interface}
\label{\detokenize{usage/classic:legacy-interface}}

\subparagraph{Run suite of scripts}
\label{\detokenize{usage/classic:run-suite-of-scripts}}
Use the \sphinxcode{\sphinxupquote{/Patent2Net/ProcessPy.bat}} or the \sphinxcode{\sphinxupquote{/Patent2Net/Process.sh}} file and enjoy!


\paragraph{Modern interface}
\label{\detokenize{usage/classic:modern-interface}}
The modern interface allows to specify a \sphinxcode{\sphinxupquote{requete.cql}} file on the command line
or by using the environment variable \sphinxcode{\sphinxupquote{P2N\_CONFIG}}.


\subparagraph{Acquire data from OPS}
\label{\detokenize{usage/classic:acquire-data-from-ops}}
Run Patent2Net:

\begin{sphinxVerbatim}[commandchars=\\\{\}]
\PYG{n}{p2n} \PYG{n}{acquire} \PYG{o}{\PYGZhy{}}\PYG{o}{\PYGZhy{}}\PYG{n}{config}\PYG{o}{=}\PYG{o}{/}\PYG{n}{path}\PYG{o}{/}\PYG{n}{to}\PYG{o}{/}\PYG{n}{RequestsSets}\PYG{o}{/}\PYG{n}{Lentille}\PYG{o}{.}\PYG{n}{cql}
\end{sphinxVerbatim}

Alternatively, you can specify the path to the \sphinxcode{\sphinxupquote{requete.cql}} using an environment variable:

\begin{sphinxVerbatim}[commandchars=\\\{\}]
export P2N\PYGZus{}CONFIG=`pwd`/RequestsSets/Lentille.cql
\end{sphinxVerbatim}

then, just run:

\begin{sphinxVerbatim}[commandchars=\\\{\}]
\PYG{c+c1}{\PYGZsh{} Acquire patent information from OPS}
\PYG{n}{p2n} \PYG{n}{acquire}

\PYG{c+c1}{\PYGZsh{} Also acquire family information for each hit}
\PYG{n}{p2n} \PYG{n}{acquire} \PYG{o}{\PYGZhy{}}\PYG{o}{\PYGZhy{}}\PYG{k}{with}\PYG{o}{\PYGZhy{}}\PYG{n}{family}
\end{sphinxVerbatim}


\subparagraph{Analyze information}
\label{\detokenize{usage/classic:analyze-information}}
When running the analysis commands like this, you should set
the \sphinxcode{\sphinxupquote{P2N\_CONFIG}} environment variable for convenience, like described above.

E.g., run:

\begin{sphinxVerbatim}[commandchars=\\\{\}]
\PYG{c+c1}{\PYGZsh{} Build all world maps}
\PYG{n}{p2n} \PYG{n}{maps}

\PYG{c+c1}{\PYGZsh{} Build all network graphs}
\PYG{n}{p2n} \PYG{n}{networks}

\PYG{c+c1}{\PYGZsh{} p2n \PYGZob{}maps,networks,tables,bibfile,iramuteq,freeplane,carrot\PYGZcb{}}
\PYG{c+c1}{\PYGZsh{} see full list below or run ``p2n \PYGZhy{}\PYGZhy{}help``}
\end{sphinxVerbatim}


\bigskip\hrule\bigskip



\paragraph{Synopsis}
\label{\detokenize{usage/classic:synopsis}}

\subparagraph{Output of “\sphinxstyleliteralintitle{\sphinxupquote{p2n \sphinxhyphen{}\sphinxhyphen{}help}}”}
\label{\detokenize{usage/classic:output-of-p2n-help}}
\begin{sphinxVerbatim}[commandchars=\\\{\}]
\PYGZdl{} p2n \PYGZhy{}\PYGZhy{}help

\PYGZhy{}\PYGZhy{}\PYGZhy{}\PYGZhy{}\PYGZhy{}\PYGZhy{}\PYGZhy{}\PYGZhy{}\PYGZhy{}\PYGZhy{}\PYGZhy{}\PYGZhy{}
Classic mode
\PYGZhy{}\PYGZhy{}\PYGZhy{}\PYGZhy{}\PYGZhy{}\PYGZhy{}\PYGZhy{}\PYGZhy{}\PYGZhy{}\PYGZhy{}\PYGZhy{}\PYGZhy{}
  p2n ops init                          Initialize Patent2Net with OPS OAuth credentials
  p2n run                               Run data acquisition and all formatters
  p2n acquire                           Run document acquisition
    \PYGZhy{}\PYGZhy{}with\PYGZhy{}family                       Also run family data acquisition with \PYGZdq{}p2n acquire\PYGZdq{}
  p2n maps                              Build maps of country coverage of patents, as well as applicants and inventors
  p2n networks                          Build various artefacts for data exploration based on network graphs
  p2n tables                            Export various artefacts for tabular data exploration
  p2n bibfile                           Export data in bibfile format
  p2n iramuteq                          Fetch more data and export it to suitable format for using in Iramuteq
  p2n freeplane                         Build mind map for Freeplane
  p2n carrot                            Export data to XML suitable for using in Carrot
  p2n images                            Fetch images and build thumbnails
  p2n interface                         Build main Patent2Net html interface

Options:
  \PYGZhy{}\PYGZhy{}config=\PYGZlt{}config\PYGZgt{}                     Path to requete.cql. Will fall back to environment variable \PYGZdq{}P2N\PYGZus{}CONFIG\PYGZdq{}.

Examples:

  \PYGZsh{} Initialize Patent2Net with OPS OAuth credentials
  p2n ops init \PYGZhy{}\PYGZhy{}key=ScirfedyifJiashwOckNoupNecpainLo \PYGZhy{}\PYGZhy{}secret=degTefyekDevgew1

  \PYGZsh{} Run query and gather data
  p2n acquire \PYGZhy{}\PYGZhy{}config=/path/to/RequestsSets/Lentille.cql \PYGZhy{}\PYGZhy{}with\PYGZhy{}family

  \PYGZsh{} Build all world maps
  export P2N\PYGZus{}CONFIG=/path/to/RequestsSets/Lentille.cql
  p2n maps

  \PYGZsh{} Run data acquisition and all targets
  p2n run
\end{sphinxVerbatim}


\subsubsection{Ad\sphinxhyphen{}hoc mode}
\label{\detokenize{usage/adhoc:ad-hoc-mode}}\label{\detokenize{usage/adhoc::doc}}
The ad\sphinxhyphen{}hoc mode allows to specify the query expression on the command line.
This gives Patent2Net a more interactive mode of operation.

\begin{sphinxShadowBox}
\begin{itemize}
\item {} 
\phantomsection\label{\detokenize{usage/adhoc:id1}}{\hyperref[\detokenize{usage/adhoc:data-dumpers}]{\sphinxcrossref{Data dumpers}}}

\item {} 
\phantomsection\label{\detokenize{usage/adhoc:id2}}{\hyperref[\detokenize{usage/adhoc:data-formatters}]{\sphinxcrossref{Data formatters}}}

\item {} 
\phantomsection\label{\detokenize{usage/adhoc:id3}}{\hyperref[\detokenize{usage/adhoc:synopsis}]{\sphinxcrossref{Synopsis}}}

\end{itemize}
\end{sphinxShadowBox}


\bigskip\hrule\bigskip



\paragraph{Data dumpers}
\label{\detokenize{usage/adhoc:data-dumpers}}
These actions run the query against OPS and display decoded/polished information in JSON format.

You might want to have a look at the \DUrole{xref,std,std-ref}{jq} documentation for working with JSON data on the command line.


\subparagraph{Full bibliographic data}
\label{\detokenize{usage/adhoc:full-bibliographic-data}}
Display bibliographic data for given query expression:

\begin{sphinxVerbatim}[commandchars=\\\{\}]
\PYG{n}{p2n} \PYG{n}{adhoc} \PYG{n}{dump} \PYG{o}{\PYGZhy{}}\PYG{o}{\PYGZhy{}}\PYG{n}{expression}\PYG{o}{=}\PYG{l+s+s1}{\PYGZsq{}}\PYG{l+s+s1}{TA=lentille}\PYG{l+s+s1}{\PYGZsq{}}
\end{sphinxVerbatim}

Expand each hit to its whole family:

\begin{sphinxVerbatim}[commandchars=\\\{\}]
\PYG{n}{p2n} \PYG{n}{adhoc} \PYG{n}{dump} \PYG{o}{\PYGZhy{}}\PYG{o}{\PYGZhy{}}\PYG{n}{expression}\PYG{o}{=}\PYG{l+s+s1}{\PYGZsq{}}\PYG{l+s+s1}{TA=lentille}\PYG{l+s+s1}{\PYGZsq{}} \PYG{o}{\PYGZhy{}}\PYG{o}{\PYGZhy{}}\PYG{k}{with}\PYG{o}{\PYGZhy{}}\PYG{n}{family}
\end{sphinxVerbatim}

Enrich each document with its register information:

\begin{sphinxVerbatim}[commandchars=\\\{\}]
\PYG{n}{p2n} \PYG{n}{adhoc} \PYG{n}{dump} \PYG{o}{\PYGZhy{}}\PYG{o}{\PYGZhy{}}\PYG{n}{expression}\PYG{o}{=}\PYG{l+s+s1}{\PYGZsq{}}\PYG{l+s+s1}{TA=lentille}\PYG{l+s+s1}{\PYGZsq{}} \PYG{o}{\PYGZhy{}}\PYG{o}{\PYGZhy{}}\PYG{k}{with}\PYG{o}{\PYGZhy{}}\PYG{n}{family} \PYG{o}{\PYGZhy{}}\PYG{o}{\PYGZhy{}}\PYG{k}{with}\PYG{o}{\PYGZhy{}}\PYG{n}{register}
\end{sphinxVerbatim}

For dumping data using the legacy Patent2Net brevet format:

\begin{sphinxVerbatim}[commandchars=\\\{\}]
\PYG{n}{p2n} \PYG{n}{adhoc} \PYG{n}{dump} \PYG{o}{\PYGZhy{}}\PYG{o}{\PYGZhy{}}\PYG{n}{expression}\PYG{o}{=}\PYG{l+s+s1}{\PYGZsq{}}\PYG{l+s+s1}{TA=lentille}\PYG{l+s+s1}{\PYGZsq{}} \PYG{o}{\PYGZhy{}}\PYG{o}{\PYGZhy{}}\PYG{n+nb}{format}\PYG{o}{=}\PYG{n}{brevet}
\end{sphinxVerbatim}


\subparagraph{Single bibliographic data field}
\label{\detokenize{usage/adhoc:single-bibliographic-data-field}}
Display list of publication numbers for given query expression:

\begin{sphinxVerbatim}[commandchars=\\\{\}]
\PYG{n}{p2n} \PYG{n}{adhoc} \PYG{n+nb}{list} \PYG{o}{\PYGZhy{}}\PYG{o}{\PYGZhy{}}\PYG{n}{expression}\PYG{o}{=}\PYG{l+s+s1}{\PYGZsq{}}\PYG{l+s+s1}{TA=lentille}\PYG{l+s+s1}{\PYGZsq{}}
\PYG{n}{p2n} \PYG{n}{adhoc} \PYG{n+nb}{list} \PYG{o}{\PYGZhy{}}\PYG{o}{\PYGZhy{}}\PYG{n}{expression}\PYG{o}{=}\PYG{l+s+s1}{\PYGZsq{}}\PYG{l+s+s1}{TA=lentille}\PYG{l+s+s1}{\PYGZsq{}} \PYG{o}{\PYGZhy{}}\PYG{o}{\PYGZhy{}}\PYG{k}{with}\PYG{o}{\PYGZhy{}}\PYG{n}{family}
\end{sphinxVerbatim}

Display list of application numbers in epodoc format:

\begin{sphinxVerbatim}[commandchars=\\\{\}]
\PYG{n}{p2n} \PYG{n}{adhoc} \PYG{n+nb}{list} \PYG{o}{\PYGZhy{}}\PYG{o}{\PYGZhy{}}\PYG{n}{expression}\PYG{o}{=}\PYG{l+s+s1}{\PYGZsq{}}\PYG{l+s+s1}{TA=lentille}\PYG{l+s+s1}{\PYGZsq{}} \PYG{o}{\PYGZhy{}}\PYG{o}{\PYGZhy{}}\PYG{n}{field}\PYG{o}{=}\PYG{l+s+s1}{\PYGZsq{}}\PYG{l+s+s1}{application\PYGZus{}number\PYGZus{}epodoc}\PYG{l+s+s1}{\PYGZsq{}}
\end{sphinxVerbatim}

Display list of \sphinxcode{\sphinxupquote{register.status}} values:

\begin{sphinxVerbatim}[commandchars=\\\{\}]
\PYG{n}{p2n} \PYG{n}{adhoc} \PYG{n+nb}{list} \PYG{o}{\PYGZhy{}}\PYG{o}{\PYGZhy{}}\PYG{n}{expression}\PYG{o}{=}\PYG{l+s+s1}{\PYGZsq{}}\PYG{l+s+s1}{TA=lentille}\PYG{l+s+s1}{\PYGZsq{}} \PYG{o}{\PYGZhy{}}\PYG{o}{\PYGZhy{}}\PYG{k}{with}\PYG{o}{\PYGZhy{}}\PYG{n}{family} \PYG{o}{\PYGZhy{}}\PYG{o}{\PYGZhy{}}\PYG{k}{with}\PYG{o}{\PYGZhy{}}\PYG{n}{register} \PYG{o}{\PYGZhy{}}\PYG{o}{\PYGZhy{}}\PYG{n}{field}\PYG{o}{=}\PYG{l+s+s1}{\PYGZsq{}}\PYG{l+s+s1}{register.status}\PYG{l+s+s1}{\PYGZsq{}}
\end{sphinxVerbatim}

\begin{sphinxadmonition}{note}{Note:}
You can use all fields available in the OPSExchangeDocument data model.
\end{sphinxadmonition}


\bigskip\hrule\bigskip



\paragraph{Data formatters}
\label{\detokenize{usage/adhoc:data-formatters}}
Generate data for world maps using d3plus/geo\_map (JSON):

\begin{sphinxVerbatim}[commandchars=\\\{\}]
\PYG{n}{p2n} \PYG{n}{adhoc} \PYG{n}{worldmap} \PYG{o}{\PYGZhy{}}\PYG{o}{\PYGZhy{}}\PYG{n}{expression}\PYG{o}{=}\PYG{l+s+s1}{\PYGZsq{}}\PYG{l+s+s1}{TA=lentille}\PYG{l+s+s1}{\PYGZsq{}} \PYG{o}{\PYGZhy{}}\PYG{o}{\PYGZhy{}}\PYG{n}{country}\PYG{o}{\PYGZhy{}}\PYG{n}{field}\PYG{o}{=}\PYG{l+s+s1}{\PYGZsq{}}\PYG{l+s+s1}{country}\PYG{l+s+s1}{\PYGZsq{}}
\PYG{n}{p2n} \PYG{n}{adhoc} \PYG{n}{worldmap} \PYG{o}{\PYGZhy{}}\PYG{o}{\PYGZhy{}}\PYG{n}{expression}\PYG{o}{=}\PYG{l+s+s1}{\PYGZsq{}}\PYG{l+s+s1}{TA=lentille}\PYG{l+s+s1}{\PYGZsq{}} \PYG{o}{\PYGZhy{}}\PYG{o}{\PYGZhy{}}\PYG{n}{country}\PYG{o}{\PYGZhy{}}\PYG{n}{field}\PYG{o}{=}\PYG{l+s+s1}{\PYGZsq{}}\PYG{l+s+s1}{country}\PYG{l+s+s1}{\PYGZsq{}} \PYG{o}{\PYGZhy{}}\PYG{o}{\PYGZhy{}}\PYG{k}{with}\PYG{o}{\PYGZhy{}}\PYG{n}{family}
\PYG{n}{p2n} \PYG{n}{adhoc} \PYG{n}{worldmap} \PYG{o}{\PYGZhy{}}\PYG{o}{\PYGZhy{}}\PYG{n}{expression}\PYG{o}{=}\PYG{l+s+s1}{\PYGZsq{}}\PYG{l+s+s1}{TA=lentille}\PYG{l+s+s1}{\PYGZsq{}} \PYG{o}{\PYGZhy{}}\PYG{o}{\PYGZhy{}}\PYG{n}{country}\PYG{o}{\PYGZhy{}}\PYG{n}{field}\PYG{o}{=}\PYG{l+s+s1}{\PYGZsq{}}\PYG{l+s+s1}{applicants}\PYG{l+s+s1}{\PYGZsq{}}
\PYG{n}{p2n} \PYG{n}{adhoc} \PYG{n}{worldmap} \PYG{o}{\PYGZhy{}}\PYG{o}{\PYGZhy{}}\PYG{n}{expression}\PYG{o}{=}\PYG{l+s+s1}{\PYGZsq{}}\PYG{l+s+s1}{TA=lentille}\PYG{l+s+s1}{\PYGZsq{}} \PYG{o}{\PYGZhy{}}\PYG{o}{\PYGZhy{}}\PYG{n}{country}\PYG{o}{\PYGZhy{}}\PYG{n}{field}\PYG{o}{=}\PYG{l+s+s1}{\PYGZsq{}}\PYG{l+s+s1}{inventors}\PYG{l+s+s1}{\PYGZsq{}}
\PYG{n}{p2n} \PYG{n}{adhoc} \PYG{n}{worldmap} \PYG{o}{\PYGZhy{}}\PYG{o}{\PYGZhy{}}\PYG{n}{expression}\PYG{o}{=}\PYG{l+s+s1}{\PYGZsq{}}\PYG{l+s+s1}{TA=lentille}\PYG{l+s+s1}{\PYGZsq{}} \PYG{o}{\PYGZhy{}}\PYG{o}{\PYGZhy{}}\PYG{n}{country}\PYG{o}{\PYGZhy{}}\PYG{n}{field}\PYG{o}{=}\PYG{l+s+s1}{\PYGZsq{}}\PYG{l+s+s1}{register.designated\PYGZus{}states}\PYG{l+s+s1}{\PYGZsq{}} \PYG{o}{\PYGZhy{}}\PYG{o}{\PYGZhy{}}\PYG{k}{with}\PYG{o}{\PYGZhy{}}\PYG{n}{register}
\end{sphinxVerbatim}

Generate data suitable for PivotTable.js (JSON):

\begin{sphinxVerbatim}[commandchars=\\\{\}]
\PYG{n}{p2n} \PYG{n}{adhoc} \PYG{n}{pivot} \PYG{o}{\PYGZhy{}}\PYG{o}{\PYGZhy{}}\PYG{n}{expression}\PYG{o}{=}\PYG{l+s+s1}{\PYGZsq{}}\PYG{l+s+s1}{TA=lentille}\PYG{l+s+s1}{\PYGZsq{}} \PYG{o}{\PYGZhy{}}\PYG{o}{\PYGZhy{}}\PYG{k}{with}\PYG{o}{\PYGZhy{}}\PYG{n}{family}
\end{sphinxVerbatim}


\bigskip\hrule\bigskip



\paragraph{Synopsis}
\label{\detokenize{usage/adhoc:synopsis}}

\subparagraph{Output of “\sphinxstyleliteralintitle{\sphinxupquote{p2n \sphinxhyphen{}\sphinxhyphen{}help}}”}
\label{\detokenize{usage/adhoc:output-of-p2n-help}}
\begin{sphinxVerbatim}[commandchars=\\\{\}]
\PYGZdl{} p2n \PYGZhy{}\PYGZhy{}help

\PYGZhy{}\PYGZhy{}\PYGZhy{}\PYGZhy{}\PYGZhy{}\PYGZhy{}\PYGZhy{}\PYGZhy{}\PYGZhy{}\PYGZhy{}\PYGZhy{}
Ad hoc mode
\PYGZhy{}\PYGZhy{}\PYGZhy{}\PYGZhy{}\PYGZhy{}\PYGZhy{}\PYGZhy{}\PYGZhy{}\PYGZhy{}\PYGZhy{}\PYGZhy{}
  p2n ops init                          Initialize Patent2Net with OPS OAuth credentials
  p2n adhoc search                      Display search results for given query expression in raw OPS format (JSON)
  p2n adhoc dump                        Display full results for given query expression in OpsExchangeDocument or Patent2NetBrevet format (JSON)
  p2n adhoc list                        Display list of values from single field for given query expression
  p2n adhoc worldmap                    Generate world map for given query expression over given field
  p2n adhoc pivot                       Generate data for pivot table

Options:
  \PYGZhy{}\PYGZhy{}expression=\PYGZlt{}expression\PYGZgt{}             Search expression in CQL format, e.g. \PYGZdq{}TA=lentille\PYGZdq{}
  \PYGZhy{}\PYGZhy{}format=\PYGZlt{}format\PYGZgt{}                     Control output format for \PYGZdq{}p2n adhoc dump\PYGZdq{},
                                        Choose from \PYGZdq{}ops\PYGZdq{} or \PYGZdq{}brevet\PYGZdq{} [default: ops].
  \PYGZhy{}\PYGZhy{}field=\PYGZlt{}field\PYGZgt{}                       Which field name to use with \PYGZdq{}p2n adhoc list\PYGZdq{} [default: document\PYGZus{}number].
  \PYGZhy{}\PYGZhy{}with\PYGZhy{}register                       Also acquire register information for each result hit.
                                        Required for \PYGZdq{}\PYGZhy{}\PYGZhy{}country\PYGZhy{}field=register.designated\PYGZus{}states\PYGZdq{}.
  \PYGZhy{}\PYGZhy{}country\PYGZhy{}field=\PYGZlt{}country\PYGZhy{}field\PYGZgt{}       Field name of country code for \PYGZdq{}p2n adhoc worldmap\PYGZdq{}
                                        e.g. \PYGZdq{}country\PYGZdq{}, \PYGZdq{}applicants\PYGZdq{}, \PYGZdq{}inventors\PYGZdq{}, \PYGZdq{}register.designated\PYGZus{}states\PYGZdq{}

Examples:

  \PYGZsh{} Initialize Patent2Net with OPS OAuth credentials
  p2n ops init \PYGZhy{}\PYGZhy{}key=ScirfedyifJiashwOckNoupNecpainLo \PYGZhy{}\PYGZhy{}secret=degTefyekDevgew1

  \PYGZsh{} Run query and output results in OpsExchangeDocument format (JSON)
  p2n adhoc dump \PYGZhy{}\PYGZhy{}expression=\PYGZsq{}TA=lentille\PYGZsq{}

  \PYGZsh{} Run query and output results in Patent2NetBrevet format (JSON)
  p2n adhoc dump \PYGZhy{}\PYGZhy{}expression=\PYGZsq{}TA=lentille\PYGZsq{} \PYGZhy{}\PYGZhy{}format=brevet

  \PYGZsh{} Run query and output list of document numbers, including family members (JSON)
  p2n adhoc list \PYGZhy{}\PYGZhy{}expression=\PYGZsq{}TA=lentille\PYGZsq{} \PYGZhy{}\PYGZhy{}with\PYGZhy{}family

  \PYGZsh{} Run query and output list of application numbers in epodoc format
  p2n adhoc list \PYGZhy{}\PYGZhy{}expression=\PYGZsq{}TA=lentille\PYGZsq{} \PYGZhy{}\PYGZhy{}field=\PYGZsq{}application\PYGZus{}number\PYGZus{}epodoc\PYGZsq{}

  \PYGZsh{} Generate data for world maps using d3plus/geo\PYGZus{}map (JSON)
  p2n adhoc worldmap \PYGZhy{}\PYGZhy{}expression=\PYGZsq{}TA=lentille\PYGZsq{} \PYGZhy{}\PYGZhy{}country\PYGZhy{}field=\PYGZsq{}country\PYGZsq{}
  p2n adhoc worldmap \PYGZhy{}\PYGZhy{}expression=\PYGZsq{}TA=lentille\PYGZsq{} \PYGZhy{}\PYGZhy{}country\PYGZhy{}field=\PYGZsq{}applicants\PYGZsq{}
  p2n adhoc worldmap \PYGZhy{}\PYGZhy{}expression=\PYGZsq{}TA=lentille\PYGZsq{} \PYGZhy{}\PYGZhy{}country\PYGZhy{}field=\PYGZsq{}inventors\PYGZsq{}
  p2n adhoc worldmap \PYGZhy{}\PYGZhy{}expression=\PYGZsq{}TA=lentille\PYGZsq{} \PYGZhy{}\PYGZhy{}country\PYGZhy{}field=\PYGZsq{}register.designated\PYGZus{}states\PYGZsq{} \PYGZhy{}\PYGZhy{}with\PYGZhy{}register

  \PYGZsh{} Generate data suitable for PivotTable.js (JSON)
  p2n adhoc pivot \PYGZhy{}\PYGZhy{}expression=\PYGZsq{}TA=lentille\PYGZsq{} \PYGZhy{}\PYGZhy{}with\PYGZhy{}family
\end{sphinxVerbatim}


\chapter{Development}
\label{\detokenize{index:development}}
../CHANGES.rst


\section{Contributing}
\label{\detokenize{contributing:contributing}}\label{\detokenize{contributing::doc}}
For contributing to the project, this is the way to go.


\subsection{Get source code}
\label{\detokenize{contributing:get-source-code}}
\begin{sphinxVerbatim}[commandchars=\\\{\}]
\PYG{n}{git} \PYG{n}{clone} \PYG{n}{https}\PYG{p}{:}\PYG{o}{/}\PYG{o}{/}\PYG{n}{github}\PYG{o}{.}\PYG{n}{com}\PYG{o}{/}\PYG{n}{Patent2net}\PYG{o}{/}\PYG{n}{P2N}\PYG{o}{\PYGZhy{}}\PYG{n}{v3}\PYG{o}{.}\PYG{n}{git}
\PYG{n}{cd} \PYG{n}{P2N}
\PYG{n}{git} \PYG{n}{checkout} \PYG{n}{develop}
\end{sphinxVerbatim}


\subsection{Create virtualenv}
\label{\detokenize{contributing:create-virtualenv}}
\begin{sphinxVerbatim}[commandchars=\\\{\}]
\PYG{n}{virtualenv} \PYG{o}{\PYGZhy{}}\PYG{o}{\PYGZhy{}}\PYG{n}{no}\PYG{o}{\PYGZhy{}}\PYG{n}{site}\PYG{o}{\PYGZhy{}}\PYG{n}{packages} \PYG{o}{.}\PYG{n}{venv36}
\PYG{n}{source} \PYG{o}{.}\PYG{n}{venv36}\PYG{o}{/}\PYG{n+nb}{bin}\PYG{o}{/}\PYG{n}{activate}
\end{sphinxVerbatim}


\subsection{Setup in development mode}
\label{\detokenize{contributing:setup-in-development-mode}}
\begin{sphinxVerbatim}[commandchars=\\\{\}]
\PYG{n}{python} \PYG{n}{setup}\PYG{o}{.}\PYG{n}{py} \PYG{n}{develop}
\end{sphinxVerbatim}


\section{Documentation}
\label{\detokenize{documentation:documentation}}\label{\detokenize{documentation::doc}}

\subsection{Introduction}
\label{\detokenize{documentation:introduction}}
Beautiful static HTML documentation can be easily built using the Sphinx documentation generator.
Sphinx uses reStructuredText as its markup language, and many of its strengths come from the power
and straightforwardness of reStructuredText.

The documentation can be built locally and also will be published to \sphinxurl{https://docs.ip-tools.org/patent2net/}.
It could also be pushed to \sphinxurl{https://readthedocs.org/}.


\subsection{Usage}
\label{\detokenize{documentation:usage}}

\subsubsection{On Linux}
\label{\detokenize{documentation:on-linux}}
Build HTML:

\begin{sphinxVerbatim}[commandchars=\\\{\}]
\PYG{n}{make} \PYG{n}{docs}\PYG{o}{\PYGZhy{}}\PYG{n}{html}
\end{sphinxVerbatim}

Display:

\begin{sphinxVerbatim}[commandchars=\\\{\}]
\PYG{n+nb}{open} \PYG{n}{doc}\PYG{o}{/}\PYG{n}{\PYGZus{}build}\PYG{o}{/}\PYG{n}{html}\PYG{o}{/}\PYG{n}{index}\PYG{o}{.}\PYG{n}{html}
\end{sphinxVerbatim}


\subsubsection{On Windows}
\label{\detokenize{documentation:on-windows}}
Build HTML:

\begin{sphinxVerbatim}[commandchars=\\\{\}]
\PYG{n}{cd} \PYG{n}{doc}
\PYG{n}{make}\PYG{o}{.}\PYG{n}{bat}
\end{sphinxVerbatim}

Display:

\begin{sphinxVerbatim}[commandchars=\\\{\}]
\PYG{n+nb}{open} \PYG{n}{\PYGZus{}build}\PYG{o}{/}\PYG{n}{html}\PYG{o}{/}\PYG{n}{index}\PYG{o}{.}\PYG{n}{html}
\end{sphinxVerbatim}


\section{Release}
\label{\detokenize{release:release}}\label{\detokenize{release:id1}}\label{\detokenize{release::doc}}

\subsection{Prerequisites}
\label{\detokenize{release:prerequisites}}
Run once to prepare the sandbox environment for release tasks:

\begin{sphinxVerbatim}[commandchars=\\\{\}]
\PYG{n}{make} \PYG{n}{setup}\PYG{o}{\PYGZhy{}}\PYG{n}{release}
\end{sphinxVerbatim}


\subsection{Cut a new release}
\label{\detokenize{release:cut-a-new-release}}
This will bump the version in setup.py, tag the repository and push to git origin.

Use from inside repository, with virtualenv activated.

\begin{sphinxVerbatim}[commandchars=\\\{\}]
\PYG{c+c1}{\PYGZsh{} Possible increments: major, minor, patch, dev}
\PYG{n}{make} \PYG{n}{release} \PYG{n}{bump}\PYG{o}{=}\PYG{n}{dev}
\end{sphinxVerbatim}


\section{Todo list and ideas}
\label{\detokenize{todo:todo-list-and-ideas}}\label{\detokenize{todo::doc}}

\subsection{Agenda}
\label{\detokenize{todo:agenda}}

\subsubsection{User UX improvements}
\label{\detokenize{todo:user-ux-improvements}}\begin{itemize}
\item {} 
Use flask and template to redesing the whole interface keeping in ming kibana and ES new capability

\item {} 
improve progress bar

\end{itemize}


\subsubsection{Improvements}
\label{\detokenize{todo:improvements}}
Although Patent2Net is fully operational, works fine and is enough to begin using Patent Information, a lot can be done to improve analysis:
\begin{itemize}
\item {} 
Correct the issues (continuous process, of course)

\item {} 
Add some more information in the result html page (ModeleContenuIndex.html). Great to add the processing date (thus can be different from gathering) and P2N version

\item {} 
As information analysis do not always represent the whole Patent Universe (i.e. french abstract) provide the proportion of P.U. concerned by each analysis

\item {} 
Treat Designated State(s) information for EP and WO patentes to complete the attractivity maps

\item {} 
Improve the Mindmap option to get it more efficient for creativity (Celso is working on)

\item {} 
Build the entire network as a gephi file for download to let new combined network analysis possible

\item {} 
Use the list of standardised applicant names from EPO to normalize nets and tables. See: {[}CSV datafile{]} (\sphinxurl{http://documents.epo.org/projects/babylon/rawdata.nsf/0/71DE2EB24A084A19C1257F3B0032BA98/})

\end{itemize}


\subsubsection{New capabilities}
\label{\detokenize{todo:new-capabilities}}
Add some new capabilities to Patent2Net, i.e.:
\begin{itemize}
\item {} 
Within the Patent Universe, build a drawings gallery with hyperlink to the Espacenet patent (Andre is working on)

\item {} 
Within the Familly Patent Universe, provide all the same analysis as with the Patent Universe (Roberto is working on)

\item {} 
Include the treatment of the Cooperative Patent Classification (CPC) with the proportion of P.U. concerned (\sphinxurl{http://www.cooperativepatentclassification.org/Archive.html})

\item {} 
Build a small database to display results of a specific (Familly) Patent Universe. Database could be {[}PouchDB{]} (\sphinxurl{https://pouchdb.com/}) or equivalent

\end{itemize}


\subsubsection{New ways for gathering and analysis}
\label{\detokenize{todo:new-ways-for-gathering-and-analysis}}
Provide some new ways of gathering and analysis of patent information, i.e.:
\begin{itemize}
\item {} 
Within the Familly Patent Universe, provide a new range of analysis, considering a familly as a unique entity (invention)

\item {} 
Limit the Familly Patent Universe to the only Priority patents, and provide a complete analysis

\item {} 
Using citations of the Familly Patent Universe, provide genealogic analysis, especially descendants to try to detect invention fronts.

\item {} 
Gather research reports when avalaible and provide analysis chains

\end{itemize}

New contributions and ideas are always welcome.


\subsection{Tasks}
\label{\detokenize{todo:tasks}}

\subsubsection{Version 3.0.0}
\label{\detokenize{todo:version-3-0-0}}

\paragraph{Completed}
\label{\detokenize{todo:completed}}\begin{itemize}
\item {} 
{[}x{]} Introduce and stabilize new data model and ad\sphinxhyphen{}hoc mode

\item {} 
{[}x{]} Write about “jq”

\item {} 
{[}x{]} Remove attr\_object\_as\_dict in favor of attr.as\_dict

\item {} 
{[}x{]} Resolve doc.designated\_states vs. doc.register.designated\_states duplication by providing a dotted name resolver for nested objects

\item {} 
{[}x{]} Refactor maps.py and tables.py to \sphinxcode{\sphinxupquote{p2n.formatter}} namespace

\item {} 
{[}x{]} “p2n adhoc dump \textendash{}format=raw” mode

\item {} 
{[}x{]} Display OPS error message when running invalid queries like “p2n adhoc dump \textendash{}expression=’pa=grohe and py=2015’”

\end{itemize}


\paragraph{Todo}
\label{\detokenize{todo:todo}}\begin{itemize}
\item {} 
{[}o{]} Kibana and elastic search usage

\item {} 
{[}o{]} Carrot2 integration

\item {} 
{[}o{]} Iramuteq processing steps

\item {} 
{[}o{]} Mindmaps integration in full web

\item {} 
{[}o{]} Network visualisation and exploring tool integration

\item {} \begin{description}
\item[{{[}o{]} OPS Register: Always sort event\sphinxhyphen{}like data in ascending order?}] \leavevmode
Right now, sort order is mixed as of “history items” vs. “actions” vs. “\{publication,application\}\_reference”.

\end{description}

\item {} 
{[}o{]} Write documentation about data model

\item {} 
{[}o{]} Complete implementation of \sphinxcode{\sphinxupquote{Patent2NetBrevet.from\_ops\_exchange\_document}} re. citations, equivalents and more

\item {} 
{[}o{]} Complete implementation of \sphinxcode{\sphinxupquote{OPSRegisterDocument}}

\item {} 
{[}o{]} Install Webhook on GitHub for automatic documentation building

\item {} 
{[}o{]} Upload pre\sphinxhyphen{}release versions to PyPI

\item {} \begin{description}
\item[{{[}o{]} Currently, python\sphinxhyphen{}epo\sphinxhyphen{}ops\sphinxhyphen{}client requires to be online because it always attempts to authenticate.}] \leavevmode
Could this be deferred to the actual first remote access to be able to work completely offline with a prewarmed cache?

\end{description}

\item {} \begin{description}
\item[{{[}o{]} Caching improvements}] \leavevmode\begin{itemize}
\item {} 
{[}o{]} Increase dogpile cache duration to one year

\item {} 
{[}o{]} Provide api/command to clear the cache

\item {} 
{[}o{]} More fine\sphinxhyphen{}grained cache ttl control

\end{itemize}

\end{description}

\item {} \begin{description}
\item[{{[}o{]} Use pyjq for providing built\sphinxhyphen{}in filtering, with raw or even named filters.}] \leavevmode
\sphinxurl{https://pypi.python.org/pypi/pyjq}

\end{description}

\item {} \begin{description}
\item[{{[}o{]} Should we compute “register.designated\_states \sphinxhyphen{} register.countries\_lapsed” to determine the actual}] \leavevmode
list of countries the patent is still valid in?

\end{description}

\end{itemize}


\section{Indices and tables}
\label{\detokenize{index:indices-and-tables}}\begin{itemize}
\item {} 
\DUrole{xref,std,std-ref}{genindex}

\item {} 
\DUrole{xref,std,std-ref}{modindex}

\item {} 
\DUrole{xref,std,std-ref}{search}

\end{itemize}



\renewcommand{\indexname}{Index}
\printindex
\end{document}